\section{Assignment}

The best way of learning is by doing. That is especially true when programming. This assignment is a big one. Mostly because you already should know most of the things about how to code in TorqueScript.

It is split up in two parts. The first one is reading comprehension. That means that you will first read a piece of code and understand how it works. First of all, this code is used only in a client. Secondly, it is about mapping keyboard buttons to functions, which will be called when pressed down and up. Read and understand what it does. It is easier than it looks.

\url{https://gist.github.com/McTwist/8bdd572fe31663d4a73ea534b5e261cd}

The second part is about taking what you know and put it to the test. You basically will take the previous function and split it into several methods. One could then use that in an object to add, remove and check binds. Extra points if you manage to be able to sort the divisions.

This assignment will take at most one month to complete.
